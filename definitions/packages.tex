\usepackage[utf8]{inputenc}                     % erlaubt Umlaute und Sonderzeichen (in Overleaf standartmäßig aktiv)
\usepackage{graphicx}                           % um Bilder einzufügen
\usepackage{tabularx}                           % um Tabellen zu erstellen
\usepackage{booktabs}                           % schönere Tabellen
\usepackage{amsmath, amsfonts, amssymb}         % Mathematische Symbole der American Mathematical Society
\usepackage[a4paper,margin=3cm]{geometry}       % um die Seitenränder zu verändern
\usepackage{physics}                            % um physikalische Einheiten zu verwenden
\usepackage{mhchem}                             % um chemische Formeln zu schreiben
\usepackage{siunitx}                            % um Einheiten zu schreiben
    \sisetup{locale = DE}                           % , statt .
    \sisetup{reset-math-version = false} % um \boldmath in siunitx zu verwenden
    \sisetup{separate-uncertainty = true}           % Unsicherheiten werden mit einem Leerzeichen getrennt
    \sisetup{per-mode = fraction}                   % Bruchstrich oder Hochzahl
    % Hier Einheiten definieren
    \DeclareSIUnit\seppi{sp}                        % Beispiel für eine neue Einheit
\usepackage[ngerman]{babel}                     % Deutsches Datumsformat, deutsches Hochkomma, etc.
\usepackage{csquotes}                           % um Anführungszeichen zu setzen
\usepackage{caption}                    
\usepackage{qrcode}                             % um QR-Codes zu erstellen
\usepackage{hyperref}                           % um Referenzen und Links Clickbarmachen
\usepackage{xcolor}                   % um Farben zu verwenden
    \hypersetup{colorlinks,                         % um Farben der hyperref Links zu setzen
        linkcolor={black},
        citecolor={black},
        urlcolor={black}}
\usepackage[style=ieee]{biblatex}               % um Literaturverzeichnisse zu erstellen
    \addbibresource{definitions/literatur.bib}      % Pfad zur .bib Datei
\usepackage{eso-pic}                            % um Bilder im Hintergrund zu platzieren
\usepackage{multicol}                           % um Text in mehreren Spalten anzuzeigen
\usepackage{float}                              % um Bilder und Tabellen an einer bestimmten Stelle zu platzieren
\usepackage{listings}                           % um Code einzufügen
\usepackage{fancyhdr}                           % um Kopf- und Fußzeilen zu erstellen
    \pagestyle{fancy}                               % um die Kopf- und Fußzeilen zu aktivieren
\usepackage{tikz}                               % um Grafiken zu erstellen
%    \usetikzlibrary{external}                       % um Tikz Grafiken zu externisieren, damit sie nicht bei jedem Kompilieren neu erstellt werden
%    \tikzexternalize[prefix=images/tikz/]                  % Pfad, in dem die externen Tikz Grafiken gespeichert werden
\usepackage{circuitikz}                         % um Schaltpläne zu erstellen
    \usetikzlibrary{arrows}                         % um Pfeile in Tikz zu erstellen
\usepackage{pgfplots}                           % um Plots zu erstellen
\usepackage[printonlyused, withpage]{acronym}       % um Abkürzungen zu erstellen, printonlyused zeigt nur die verwendeten Abkürzungen an, withpage zeigt die Seitenzahl an, auf der die Abkürzung verwendet wird; werden in acronyms.tex definiert

% Definition aus dem der Laborbericht-Vorlage

\definecolor{MSBlue}{rgb}{.204,.353,.541}
\definecolor{MSLightBlue}{rgb}{.31,.506,.741}

% Set formats for each heading level
%\titleformat*{\section}{\rmfamily\bfseries\huge\color{MSBlue}\lowercase}
%\titleformat{\section}[hang]{\rmfamily\bfseries\huge\color{MSBlue}\lowercase}{\thesection}{1em}{}[]
%\titleformat{\subsection}{\large\bfseries\sffamily\uppercase}{\thesubsection}{1em}{}
%\titleformat{\subsubsection}{\sffamily\bfseries}{\thesubsubsection}{1em}{}


% auxillary symbols
\renewcommand{\tilde}{\symbol{126}}
\newcommand{\define}{\stackrel{!}{=}}
\renewcommand{\equiv}{\,\widehat{=}\,}
\newcommand{\subsubsubsection}{\textbf}
\newcommand{\re}{\mathrm{Re}}
\newcommand{\pr}{\mathrm{Pr}}
\newcommand{\st}{\mathrm{St}}
\newcommand{\fr}{\mathrm{Fr}}
\newcommand{\nus}{\mathrm{Nu}}
\newcommand{\gr}{\mathrm{Gr}}
\newcommand{\ra}{\mathrm{Ra}}
\newcommand{\mif}{\quad\mathrm{\babel{if}{falls}}\quad}
\newcommand{\with}{\quad\mathrm{\babel{with}{mit}}\quad}
\newcommand{\for}{\quad\mathrm{\babel{for}{f"ur}}\quad}
%\renewcommand{\not}{\not}
\newcommand{\im}{i}
\newcommand{\ariwam}{ARiWaM}
\newcommand{\matlab}{MATLAB}

% format specifications
\renewcommand{\emph}{\textbf}
\newcommand{\file}{\textit}
\newcommand{\cmd}{\texttt}
\newcommand{\ten}{\boldsymbol}
%\newcommand{\unit}{\mathrm}
\newcommand{\lemma}{\textit}
\newcommand{\deutsch}[1]{german: \textit{#1}}
\renewcommand{\index}{\emph}

%\renewcommand{\labelenumi}{\alph{enumi})}

\setlength{\parindent}{0em}
\setlength{\parskip}{1.5ex plus0.5ex minus0.5ex}
\setlength{\captionmargin}{3em}

% Hurenkinder und Schusterjungen verhindern
\clubpenalty = 10000
\widowpenalty = 10000
\displaywidowpenalty = 10000
\interlinepenalty = 10000 