\usepackage[utf8]{inputenc} % erlaubt Umlaute und Sonderzeichen (in Overleaf standartmäßig aktiv)
\usepackage{graphicx} % um Bilder einzufügen
\usepackage{tabularx} % um Tabellen zu erstellen
\usepackage{amsmath, amsfonts, amssymb} % Mathematische Symbole der American Mathematical Society
\usepackage[a4paper,margin=3cm]{geometry} % um die Seitenränder zu verändern
\usepackage{siunitx} % um Einheiten zu schreiben
\usepackage[ngerman]{babel} % Deutsches Datumsformat, deutsches Hochkomma, etc.
\usepackage{csquotes} % um Anführungszeichen zu setzen
\usepackage{hyperref} % um Referenzen und Links Clickbarmachen
\usepackage{xcolor} % um Farben zu verwenden
    \hypersetup{colorlinks,
        linkcolor={black!50!black},
        citecolor={blue!50!black},
        urlcolor={blue!80!black}}
\usepackage{biblatex} % um Literaturverzeichnisse zu erstellen
    \addbibresource{definitions/literatur.bib} % Pfad zur .bib Datei
\usepackage{eso-pic} % um Bilder im Hintergrund zu platzieren
\usepackage{multicol} % um Text in mehreren Spalten anzuzeigen
\usepackage{float} % um Bilder und Tabellen an einer bestimmten Stelle zu platzieren
\usepackage{booktabs} % um Tabellen zu erstellen
\usepackage{listings} % um Code einzufügen

\usepackage{fancyhdr} % um Kopf- und Fußzeilen zu erstellen
\usepackage{tikz} % um Grafiken zu erstellen
\usepackage{circuitikz} % um Schaltpläne zu erstellen
    \usetikzlibrary{arrows} % um Pfeile in Tikz zu erstellen
%\usepackage{pgfplots} % um Plots zu erstellen
\usepackage[printonlyused, withpage]{acronym} % um Abkürzungen zu erstellen, printonlyused zeigt nur die verwendeten Abkürzungen an, withpage zeigt die Seitenzahl an, auf der die Abkürzung verwendet wird

\pagestyle{fancy}